\chapter{Introduction}
\label{ch:introduction}

\section{Preface}


\section{Structure of this thesis}

Here, we give a brief overview of the topics explored in the following chapters.
\vspace{1mm}

\subsection{Chapter~\ref{ch:introduction}: Introduction}
Abstract here for introduction
\vspace{1mm}

\subsection{Chapter~\ref{ch:adaptive}: Title here for adaptive}
Abstract here for adaptive
\vspace{1mm}

\subsection{Chapter~\ref{ch:orbitalfield}: Orbital effect of magnetic field on the Majorana phase diagram}
Studies of Majorana bound states in semiconducting nanowires frequently neglect the orbital effect of a magnetic field.
Systematically studying its role leads us to several conclusions for designing Majoranas in this system.
Specifically, we show that for experimentally relevant parameter values the orbital effect of a magnetic field has a stronger impact on the dispersion relation than the Zeeman effect.
While Majoranas do not require the presence of only one dispersion subband, we observe that the size of the Majoranas becomes unpractically large, and the band gap unpractically small, when more than one subband is filled.
Since the orbital effect of a magnetic field breaks several symmetries of the Hamiltonian, it leads to the appearance of large regions in parameter space with no band gap whenever the magnetic field is not aligned with the wire axis.
The reflection symmetry of the Hamiltonian with respect to the plane perpendicular to the wire axis guarantees that the wire stays gapped in the topologically nontrivial region as long as the field is aligned with the wire.
\vspace{1mm}

\subsection{Chapter~\ref{ch:supercurrent}: Supercurrent Interference in Few-Mode Nanowire Josephson Junctions}
Junctions created by coupling two superconductors via a semiconductor nanowire in the presence of high magnetic fields are the basis for the potential detection, fusion and braiding of Majorana bound states.
We study NbTiN/InSb nanowire/NbTiN Josephson junctions and find that the dependence of the critical current on the magnetic field exhibits gate-tunable nodes.
This is in contrast with a well-known Fraunhofer effect, under which critical current nodes form a regular pattern with a period fixed by the junction area.
Based on a realistic numerical model we conclude that the Zeeman effect induced by the magnetic field and the spin-orbit interaction in the nanowire are insufficient to explain the observed evolution of the Josephson effect.
We find the interference between the few occupied one-dimensional modes in the nanowire to be the dominant mechanism responsible for the critical current behavior.
We also report a strong  suppression of critical currents at finite magnetic fields that should be taken into account when designing circuits based on Majorana bound states.
\vspace{1mm}

\subsection{Chapter~\ref{ch:spinorbit}: Spin-Orbit Protection of Induced Superconductivity in Majorana Nanowires}
Spin-orbit interaction (SOI) plays a key role in creating Majorana zero modes in semiconductor nanowires proximity coupled to a superconductor. We track the evolution of the induced superconducting gap in InSb nanowires coupled to a NbTiN superconductor in a large range of magnetic field strengths and orientations. Based on realistic simulations of our devices, we reveal SOI with a strength of 0.15--0.35 eV\AA. Our approach identifies the direction of the spin-orbit field, which is strongly affected by the superconductor geometry and electrostatic gates.
\vspace{1mm}

\subsection{Chapter~\ref{ch:zigzag}: Title here for zigzag}
Abstract here for zigzag
\vspace{1mm}

\subsection{Chapter~\ref{ch:shortjunction}: Robustness of Majorana bound states in the short-junction limit}
We study the effects of strong coupling between a superconductor and a semiconductor nanowire on the creation of the Majorana bound states, when the quasiparticle dwell time in the normal part of the nanowire is much shorter than the inverse superconducting gap.
This ``short-junction'' limit is relevant for the recent experiments using the epitaxially grown aluminum characterized by a transparent interface with the semiconductor and a small superconducting gap.
We find that the small superconducting gap does not have a strong detrimental effect on the Majorana properties.
Specifically, both the critical magnetic field required for creating a topological phase and the size of the Majorana bound states are independent of the superconducting gap.
The critical magnetic field scales with the wire cross section, while the relative importance of the orbital and Zeeman effects of the magnetic field is controlled by the material parameters only: $g$ factor, effective electron mass, and the semiconductor-superconductor interface transparency.

\subsection{Chapter~\ref{ch:weakantilocalization}: Title here for weakantilocalization}
Abstract here for weakantilocalization
\vspace{1mm}

\cite{Beenakker1992} % XXX: it needs a citation

\references{dissertation}
