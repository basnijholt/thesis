\chapter*{Summary}
\addcontentsline{toc}{chapter}{Summary}
\setheader{Summary}

\co{Majoranas can be used to build a topological quantum computer}
By exploiting quantum mechanics, quantum computers are able to solve problems---for example in quantum chemistry---that are far out of reach of classical computers.
Conventional approaches to quantum computers use qubits (quantum bits) that can either be made out of superconducting circuits (the currently leading approach used to demonstrate quantum supremacy), store quantum information in the state of single electrons, or in several other ways.
However, all these designs share the same limitation: noise easily spoils the quantum states and therefore makes quantum error correction necessary.
A fundamentally different approach is to use exotic particles: Majorana bound states, or simply Majoranas.
These particles do not exist in nature but were predicted to exist in engineered devices.
Majorana bound states offer the advantage of being resilient against small energy fluctuations or noise due to being protected by the system's fundamental symmetry.

\co{There is an experimental search guided by a simple 1D model predicting Majoranas.}
There is an active search across many groups worldwide trying different pathways to experimentally create and detect these particles.
One promising approach to create Majoranas relies on a theoretical prediction that a one-dimensional nanowire combining superconductivity, spin-orbit coupling, a tunable chemical potential, and a magnetic field should support Majoranas.
Whenever the device parameters are in the right regime, the Majoranas should appear at the edges of the nanowire.

\co{We fix the model by including several important physical effects, and use adaptive sampling to improve our results.}
The experimental results diverge from the predictions made by the minimal theoretical model due to it missing several physical phenomena.
To model the Majorana nanowires more realistically, we include previously neglected physical effects by considering the full three-dimensional geometry of the nanowire.
These more complex models exceed the reach of analytical theories and require intensive numerical calculations instead.
To cope with the increasing computational complexity, we develop adaptive parallel sampling algorithms (discussed in the Ch.~\ref{ch:adaptive}), which in our research typically speeds up simulations by at least an order of magnitude.
Our approach is illustrated on the cover of this thesis, which shows the conductance of a Majorana nanowire with the interesting regions sampled more accurately.
The improved models and efficient sampling, allow us to unveil potential challenges that were not present in the simple model.

\co{Including the orbital effect leads us to several important considerations.}
In Ch.~\ref{ch:orbitalfield}, we study how the electrons are influenced by the magnetic field while moving across the nanowire---an effect completely neglected in the minimal model.
We observe that this effect has a stronger impact on the Majoranas than what is included in the minimal model.
Specifically, we observe that the protection of the Majoranas nearly vanishes when the electron density in the nanowire is high, and find that the magnetic field has to be precisely aligned with the nanowire to guarantee the presence of Majoranas.

\co{Studying Majorana Josephson junctions, essential for braiding, put more restrictions on the design of the system.}
Bringing two Majorana-carrying nanowires in contact and allowing a supercurrent flow between them is required for making a Majorana qubit.
In Ch.~\ref{ch:supercurrent}, we apply our numerical model to analyze experimentally observed behavior of these nanowire junctions.
While our results agree with the experimental observations, we observe that the supercurrent decreases by an order of magnitude when Majoranas appear.
This suppression poses a new challenge in creating a Majorana qubit.

\co{Using a zigzag geometry solves the problem of having a small gap.}
Our findings may seem like bad news for the creation of Majoranas---unexpected pitfalls overlooked by the simplified models.
The detailed simulations, however, bring new opportunities as well.
Not being constrained to analyzing the devices that are easy to solve, we are able to design a new \emph{zigzag} device geometry, that improves the robustness of Majoranas by an order of magnitude.
In Ch.~\ref{ch:zigzag}, we show that using a zigzag device geometry (instead of a straight nanowire) eliminates the long electron trajectories that are responsible for the degradation of the Majorana properties.
In addition to the improved robustness of the Majoranas, this new zigzag geometry is insensitive to the geometric details and device tuning.
This proposal is now the topic of active experimental investigations by several groups.

\chapter*{Samenvatting}
\addcontentsline{toc}{chapter}{Samenvatting}
\setheader{Samenvatting}
{\selectlanguage{dutch}

\co{Majorana's kunnen gebruikt worden om een topologische kwantumcomputer te bouwen}
Door gebruik te maken van kwantummechanica, kunnen kwantumcomputers problemen oplossen---bijvoorbeeld in de kwantumchemie---die ver buiten het bereik van klassieke computers liggen.
Conventioneel gebruiken kwantumcomputers qubits (quantumbits) die ofwel gemaakt kunnen worden van supergeleidende circuits (de momenteel leidende aanpak en gebruikt om kwantum suprematie aan te tonen), of van enkele elektronen waarin de kwantuminformatie in hun toestand wordt opgeslagen, of op diverse andere manieren.
Al deze ontwerpen hebben echter dezelfde beperking: ruis kan de kwantumtoestanden gemakkelijk vernietigen en daardoor is kwantum error correctie noodzakelijk.
Een fundamenteel andere aanpak is het gebruik van exotische deeltjes: gebonden Majoranatoestanden of gewoon Majorana's.
Deze deeltjes bestaan niet in de natuur, maar het is voorspeld dat ze kunnen bestaan in artificieel gemaakte apparaten.
Majorana's bieden het voordeel dat ze bestand zijn tegen kleine energiefluctuaties of ruis, omdat ze beschermd zijn door de fundamentele symmetrie van het systeem.

\co{Er is een experimentele zoektocht, aan de hand van een eenvoudig 1D-model dat Majorana's voorspelt.}
Wereldwijd wordt er in veel groepen actief gezocht naar verschillende manieren om deze deeltjes experimenteel te creëren en te detecteren.
Een veelbelovende aanpak om Majorana's te creëren, is gebaseerd op een theoretische voorspelling dat een eendimensionale nanodraad die supergeleiding, spin-baankoppeling, een regelbare chemische potentiaal, en een magnetisch veld combineert, Majorana's zou moeten kunnen bevatten.
Als de parameters in het juiste regime zijn, zouden de Majorana's aan de uiteinden van de nanodraad moeten verschijnen.

\co{We verbeteren het model door verschillende belangrijke fysische effecten mee te nemen en gebruiken adaptieve bemonstering om onze resultaten te verbeteren.}
De experimentele resultaten wijken echter af van de voorspellingen van het minimale theoretische model omdat het verschillende fysische verschijnselen mist.
Om de Majorana-nanodraden realistischer te modelleren, nemen we eerder verwaarloosde fysische effecten mee door de volledige driedimensionale geometrie van de nanodraad te beschouwen.
Deze complexere modellen gaan het bereik van analytische theorieën te boven en vereisen in plaats daarvan intensieve numerieke berekeningen.
Om met de toenemende computationele complexiteit om te gaan, ontwikkelen we adaptieve parallelle bemonsteringsalgoritmen (besproken in hoofdstuk~\ref{ch:adaptive}), die in ons onderzoek de simulaties typisch versnellen met ten minste een orde van grootte.
Onze aanpak wordt geïllustreerd op de cover van dit proefschrift.
Dit laat de geleiding door een Majorana-nanodraad zien, waar de interessante regio's nauwkeuriger zijn bemonsterd.  % XXX: papa vondt dit raar
De verbeterde modellen en efficiëntere bemonstering, stellen ons in staat om potentiële uitdagingen te onthullen die niet aanwezig waren in het eenvoudige model.

\co{Inclusief het orbitale effect leidt ons tot een aantal belangrijke overwegingen.}
In hoofdstuk~\ref{ch:orbitalfield} bestuderen we hoe de elektronen beïnvloed worden door het magnetisch veld als ze loodrecht op de nanodraad bewegen---een effect dat volledig verwaarloosd wordt in het minimale model.
We zien dat dit effect een grotere impact heeft op de Majorana's dan dat wat er in het minimale model wordt beschouwd.
Daarbij zien we dat de bescherming van de Majorana's vrijwel verdwijnt als de elektronendichtheid in de nanodraad hoog is en ontdekken we dat het magnetisch veld exact uitgelijnd moet worden met de richting van de nanodraad, om de aanwezigheid van Majorana's te garanderen.

\co{Bestudering van Majorana Josephson-juncties, essentieel voor vlechten, legt meer beperkingen op aan het ontwerp van het systeem.}
Het in contact brengen van twee Majorana-bevattende nanodraden en het toestaan van een superstroom daartussen, is een vereiste voor het maken van een Majorana-qubit.
In hoofdstuk~\ref{ch:supercurrent} passen we ons numerieke model toe om het experimenteel waargenomen gedrag van deze nanodraadjuncties te analyseren.
Hoewel onze resultaten overeenkomen met de experimentele waarnemingen, zien we dat de superstroom met een orde van grootte afneemt wanneer Majorana's verschijnen.
Deze onderdrukking van de superstroom vormt een nieuwe uitdaging voor het maken van een Majorana-qubit.

\co{Het gebruik van een zigzag-geometrie lost het probleem van een kleine opening op.}
Onze bevindingen lijken misschien slecht nieuws voor de creatie van Majorana's---onverwachte valkuilen die niet voortkomen uit vereenvoudigde modellen.
De gedetailleerde simulaties bieden echter ook nieuwe kansen.
Omdat we ons niet beperken tot het analyseren van de apparaten die gemakkelijk te bestuderen zijn, zijn we in staat om een nieuwe \emph{zigzag}-apparaatgeometrie te ontwerpen, die de robuustheid van Majorana's met een orde van grootte verbetert.
In hoofdstuk~\ref{ch:zigzag} laten we zien dat het gebruik van een zigzag-apparaatgeometrie (in plaats van een rechte nanodraad) de lange elektronenpaden elimineert, die verantwoordelijk zijn voor de verslechtering van de Majorana-eigenschappen.
Naast de verbeterde robuustheid van de Majorana's, is deze nieuwe zigzag-geometrie ongevoelig voor de geometrische details en de precieze afstellingen van het apparaat.
Ons voorstel wordt nu actief onderzocht door verschillende experimentele groepen wereldwijd.
}