\chapter*{Summary}
\addcontentsline{toc}{chapter}{Summary}
\setheader{Summary}

\co{Majoranas can be used to build a topological quantum computer}
By exploiting quantum mechanics, quantum computers are able to solve problems---for example in quantum chemistry---that are far out of reach of classical computers.
Conventional approach to quantum computers uses qubits (quantum bits) that can either be made out of superconducting circuits (the currently leading approach used to demonstrate quantum supremacy), store quantum information in the state of single electrons, or in several other ways.
However, all these designs share the same limitation: noise easily spoils the quantum states and therefore makes quantum error correction necessary.
A fundamentally different approach is to use exotic particles: Majorana bound states, or simply Majoranas.
These particles do not exist in nature but were predicted to exist in engineered devices.
Majorana bound states offer the advantage of being resilient against small energy fluctuations or noise due to being protected by the system's fundamental symmetry.

\co{There is an experimental search guided by a simple 1D model predicting Majoranas.}
There is an active search across many groups worldwide trying different pathways to experimentally create and detect these particles.
One promising approach to create Majoranas relies on a theoretical prediction that a one-dimensional nanowire combining superconductivity, spin-orbit coupling, a tunable chemical potential, and a magnetic field should support Majoranas.
Whenever the device parameters are in the right regime, the Majoranas should appear at the edges of the nanowire.

\co{We fix the model by including several important physical effects, and use adaptive sampling to improve our results.}
The experimental results diverge from the predictions of the minimal theoretical model due to it missing several physical phenomena.
To model the Majorana nanowires more realistically, we include previously neglected physical effects by considering the full three-dimensional geometry of the nanowire.
These more complex models exceed the reach of analytical theories and require intensive numerical calculations instead.
To cope with the increasing computational complexity, we developed adaptive parallel sampling algorithms (discussed in the Ch.~\ref{ch:adaptive}), which in our research typically sped up simulations by at least an order of magnitude.
Our approach is illustrated on the cover of this thesis, which shows conductance of a Majorana nanowire with the interesting regions sampled more accurately.
Using the improved models and efficient sampling, allowed us to unveil potential challenges that were not present in the simple model.

\co{Including the orbital effect leads us to serveral important considerations.}
In Ch.~\ref{ch:orbitalfield}, we study how the electrons are influenced by the magnetic field while moving across the nanowire---an effect completely neglected in the minimal model.
We observe that this effect has a stronger impact on the Majoranas than what is included in the minimal model.
Specifically, we observe that the protection of Majoranas nearly vanishes when the electron density in the nanowire is high, and find that the magnetic field has to be precisely aligned with the nanowire to guarantee the presence of Majoranas.

\co{Studying Majorana Josephson junctions, essential for braiding, put more restrictions on the design of the system.}
Bringing two Majorana-carrying nanowires in contact and allowing a supercurrent flow beteween them is required for making a Majorana qubit.
In Ch.~\ref{ch:supercurrent}, we apply our numerical model to analyze experimentally observed behavior of these nanowire junctions.
While our results agree with the experimental observations, we observe that the supercurrent decreases by an order of magnitude when Majoranas appear.
This suppression poses a new challenge in creating a Majorana qubit.

\co{Using a zigzag geometry solves the problem of having a small gap.}
Our findings may seem like bad news for the creation of Majoranas---unexpected pitfalls overlooked by the simplified models.
The detailed simulations, however, bring new opportunities as well.
Not being constrained to analyzing the devices that are easy to solve, we were able to design a new \emph{zigzag} device geometry, that improves the robustness of Majoranas by an order of magnitude.
In Ch~\ref{ch:zigzag}, we show that using a zigzag device geometry (instead of a straight nanowire) eliminates the long electron trajectories that are responsible for the degradation of Majorana properties.
In addition to the improved robustness of Majoranas, this new zigzag geometry is insensitive to the geometric details and device tuning.
This proposal is now the topic of active experimental investigations by several groups.

\chapter*{Samenvatting}
\addcontentsline{toc}{chapter}{Samenvatting}
\setheader{Samenvatting}
{\selectlanguage{dutch}

Nederlandse samenvatting hier.

}